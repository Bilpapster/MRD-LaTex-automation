\documentclass[12pt, letterpaper]{article}
\usepackage{fullpage}
\usepackage[english,greek]{babel}
\usepackage[headsep=0cm, headheight=94pt, bottom=2cm]{geometry}
\usepackage{graphicx}
\usepackage[skip=10pt plus1pt, indent=0pt]{parskip}
\usepackage{fancyhdr}
\usepackage{times}
\usepackage{libertine}
\usepackage[export]{adjustbox}

\begin{document}
\pagestyle{fancy}
\fancyfoot{}
\fancyhead[R]{\includegraphics[scale=0.07]{images/mrd_logo.png}}
\renewcommand{\headrulewidth}{0pt}
\normalsize


\textbf{ΓΝΩΜΗ}

\textbf{Εισηγητής:} Ρωσική ομοσπονδία

\textbf{Υποστηρικτές/τριες:} Αίγυπτος, Αργεντινή, Αυστραλία, Αυστρία, Βέλγιο, Βουλγαρία, Γαλλία, Γερμανία, Δανία, Ελλάδα, Ηνωμένο Βασίλειο, Ιαπωνία, Ισπανία, Ιταλία, Κίνα, Κύπρος, Μαρόκο, Νορβηγία, Ολλανδία, Ουγγαρία, Ουκρανία, Πορτογαλία, Ρουμανία, Ρωσική Ομοσπονδία, Σλοβακία, Σουηδία, Τουρκία, Φιλιππίνες

\textbf{Η επιτροπή «ΕΠΙΤΡΟΠΗ ΓΙΑ ΤΑ ΔΙΚΑΙΩΜΑΤΑ ΤΟΥ ΑΝΘΡΩΠΟΥ» με θέμα ημερήσιας διάταξης «ΤΟ ΔΙΚΑΙΩΜΑ ΣΤΗΝ ΕΛΕΥΘΕΡΙΑ ΤΗΣ ΕΚΦΡΑΣΗΣ»,}

\textit{Υπηρετώντας} τον σκοπό του Άρθρου 19 της Οικουμενικής Διακήρυξης Δικαιωμάτων του Ανθρώπου σχετικά με την Ελευθερία της Γνώμης και Έκφρασης,

\textit{Υπογραμμίζοντας} τη σημασία του Γενικού Σχολίου 34 του Διεθνούς Συμφώνου για τα Ατομικά και Πολιτικά Δικαιώματα,

\textit{Δεδομένης} της ανάγκης διασφάλισης της έγκαιρης και έγκυρης ενημέρωσης των πολιτών σχετικά με τα ζητήματα που ταλανίζουν την κοινωνία μας,

\textit{Αναγνωρίζοντας} τον καίριο ρόλο των μέσων κοινωνικής δικτύωσης στην καθημερινή και κοινωνική ζωή των ατόμων,

\textit{Τονίζοντας} την σπουδαιότητα την πολυφωνίας και του πλουραλισμού των μέσων ενημέρωσης,

\textit{Επισημαίνοντας} τον αυξημένο κίνδυνο που εγκυμονεί η παραπληροφόρηση και οι ψευδείς ειδήσεις στην καθημερινότητα των πολιτών,

\begin{enumerate}
  
  \item \textbf{Προτείνει} με τη χρήση τεχνητής νοημοσύνης, τη δημιουργία ενός ειδικού αλγορίθμου για τον εντοπισμό επαναλαμβανόμενων μοτίβων λέξεων και υβριστικών εκφράσεων στα διαδικτυακά μέσα κοινωνικής δικτύωσης που συνδέονται με διχαστικό και προσβλητικό λόγο.
  
  \item \textbf{Προτρέπει} τα κράτη μέρη να ενισχύσουν τη συνεργασία τους με την εκάστοτε Δίωξη του Ηλεκτρονικού Εγκλήματος, προκειμένου αυτή να δραστηριοποιηθεί στον εντοπισμό περιστατικών καταπάτησης του δικαιώματος της ελευθερίας της έκφρασης στα μέσα κοινωνικής δικτύωσης καθώς και στον αδικαιολόγητο αποκλεισμό χρηστών σε αυτά.
  
  \item \textbf{Συστήνει} στους εκπαιδευτικούς φορείς να προβούν σε κατάλληλη ενημέρωση των μαθητών ήδη από μικρή ηλικία ( μάθημα πολιτικής παιδείας, σεμινάρια, webinars, podcasts, διαφημιστικών σποτ, ομιλίες πνευματικών ανθρώπων) αναφορικά με την προστασία του δικαιώματος της ελευθερίας της έκφρασης στο διαδίκτυο.
  
  \item \textbf{Αξιολογεί θετικά} τη δια βίου μάθηση σε ζητήματα ανθρωπίνων δικαιωμάτων μέσω διεξαγωγής σχετικών επιμορφωτικών σεμιναρίων, από ΜΚΟ σε συνεργασία με τις τοπικές κοινότητες, ώστε οι πολίτες να είναι σε θέση να εντοπίζουν και να καταγγέλλουν περιστατικά παραβίασης και να οδηγούνται στην δικαιοσύνη.
  
  \item \textbf{Στηρίζει} την υιοθέτηση μηχανισμών ελέγχου από τις κρατικές αρχές όσον αναφορά στις δηλώσεις δημοσίων προσώπων και κρατικών φορέων που στοχεύουν στην δημιουργία προπαγάνδας και δημόσιας παραπλάνησης στο Διαδίκτυο.
  
  \item \textbf{Κρίνει θεμιτή} τη δημιουργία ιατρικών επιστημονικών ιστοτόπων παγκόσμιας κλίμακας υπό την Αιγίδα του ΟΗΕ και του Παγκόσμιου Οργανισμού Υγείας με τη συμμετοχή επιστημόνων διεθνούς κύρους που θα αποσκοπούν στην εξάλειψη δόλιων και άνευ επιστημονικής τεκμηρίωσης απόψεων, οι οποίες αποσκοπούν στην παραπλάνηση των πολιτών σε ευαίσθητα θέματα δημόσιας υγείας.
  
  \item \textbf{Αξιολογεί θετικά} τη δημιουργία ενός "χάρτη" δεοντολογίας δεδομένων σε παγκόσμια κλίμακα ο οποίος θα έχει ως στόχο να θέσει αυστηρά κριτήρια για τον χειρισμό των δεδομένων υγείας στο Διαδίκτυο που περιλαμβάνονται στο επιστημονικό και στατιστικό αναλυτικό ιατρικό έργο.
  
  \item \textbf{Ενθαρρύνει} την σύσταση ενός ανεξάρτητου κρατικού φορέα υπεύθυνου για τον έλεγχο, τη ρύθμιση και τον περιορισμό ζητημάτων που άπτονται της δημόσιας υγείας όταν η εκάστοτε κυβέρνηση κρίνει ότι η δημόσια υγεία κινδυνεύει από ανυπόστατες και επικίνδυνες πληροφορίες.
  
  \item \textbf{Υποστηρίζει} την άμεση πληροφόρηση και προειδοποίηση των χρηστών σε πλατφόρμες μέσων κοινωνικής δικτύωσης για θέματα που περιέχουν παραπλανητικές πληροφορίες ή ανυπόστατα στοιχεία σχετικά με την δημόσια υγεία όπως θα αξιολογούνται από επιστήμονες καθώς και από τον Παγκόσμιο Οργανισμό Υγείας.
  
  \item \textbf{Ενθαρρύνει} τα κράτη μέρη να ενισχύσουν τον ανθρωποκεντρικό χαρακτήρα των μαθημάτων στο πλαίσιο του σχολείου, μέσω της ενίσχυσης του διαλόγου, με σκοπό την καλλιέργεια της ανεκτικότητας και την ευαισθητοποίηση των μαθητών σχετικά με τα ζητήματα αποκλεισμού του δικαιώματος της ελευερίας έκφρασης των ευάλωτων κοινωνικών ομάδων.
  
  \item \textbf{Συνιστά} τη στήριξη από τα κράτη μέρη τοπικών και περιφερειακών ιδιωτικών φορέων ενημέρωσης που εστιάζουν στην εκπροσώπηση διαφορετικών κοινωνικών ομάδων με σκοπό την ενίσχυση του πλουραλισμού και την αποφυγή του μονοπωλίου.
  
  \item \textbf{Κρίνει απαραίτητο} τον έλεγχο από ανεξάρτητους φορείς της τήρησης των κανόνων δημοσιογραφικής δεοντολογίας, ώστε να εξασφαλίζεται ο σεβασμός και η συμμετοχή όλων των κοινωνικών ομάδων, με σκοπό την αποφυγή υπηρέτησης εξω-δημοσιογραφικών συμφερόντων.
  
  \item \textbf{Παροτρύνει} την δραστηριοποίηση ανεξάρτητων δημοσιογραφικών φορέων και ΜΚΟ ( επι παραδείγματι \textlatin{RCEF} ) με σκοπό την διασφάλιση της ανεξάρτητης και απαιτούμενης πλουραλιστικής ενημέρωσης και πληροφόρησης του κοινού.
  
  \item \textbf{Υποστηρίζει} την σύσταση μίας ανεξάρτητης αρχής σε κάθε κράτος, η οποία θα διεξάγει συστηματικούς ελέγχους σε περιπτώσεις αφαίρεσης περιεχομένου από διαδικτυακές πλατφόρμες, ώστε να διασφαλιστεί η ορθή εφαρμογή των κριτηρίων που έχουν θεσπισθεί.
  
\end{enumerate}


\end{document}