\documentclass[12pt, letterpaper]{article}
\usepackage{fullpage}
\usepackage[english,greek]{babel}
\usepackage[headsep=0cm, headheight=94pt, bottom=2cm]{geometry}
\usepackage{graphicx}
\usepackage[skip=10pt plus1pt, indent=0pt]{parskip}
\usepackage{fancyhdr}
\usepackage{times}
\usepackage{libertine}
\usepackage[export]{adjustbox}

\begin{document}
\pagestyle{fancy}
\fancyfoot{}
\fancyhead[R]{\includegraphics[scale=0.07]{images/mrd_logo.png}}
\renewcommand{\headrulewidth}{0pt}
\normalsize


\textbf{ΓΝΩΜΗ}

\textbf{Εισηγητής:} Γερμανία

\textbf{Υποστηρικτές/τριες:} Αργεντινή, Αίγυπτος, Αυστραλία, Αυστρία, Βέλγιο, Βενεζουέλα, Βουλγαρία, Γαλλία, Γερμανία, Δανία, Ελλάδα, Κίνα, Ιαπωνία, Ινδία, Ηνωμένο Βασίλειο, Ιταλία, Ισπανία, Κύπρος, Μαρόκο, Μεξικό, Νορβηγία, Νότια Αφρική, Ολλανδία, Ουγγαρία, Ουκρανία, Πορτογαλία, Ρουμανία, Σλοβακία, Σουηδία, Τουρκία

\textbf{Η επιτροπή «Επιτροπή για τα Δικαιώματα του Ανθρώπου» με θέμα ημερήσιας διάταξης «Το Δικαίωμα στην Ελευθερία Σκέψης, Συνείδησης και Θρησκείας»,}

\textit{Σεβόμενη} το άρθρο 18 και 27 του Διεθνούς Συμφώνου για τα Οικονομικά, Κοινωνικά και Πολιτιστικά Δικαιώματα (1976),

\textit{Αναλογιζόμενη} το Γενικό Σχόλιο 22 και 23 της Επιτροπής για τα Δικαιώματα του Ανθρώπου (1993 και 1994),

\textit{Υπηρετώντας} τον σκοπό των Ηνωμένων Εθνών όπως καταγράφεται στο άρθρο 1 παρ. 3 του καταστατικού του Χάρτη για την ανάπτυξη και ενθάρρυνση του σεβασμού των ανθρωπίνων δικαιωμάτων χωρίς διάκριση φυλής, φύλου, γλώσσας ή θρησκείας,

\textit{Εμπνεόμενη} από τις σύγχρονες προσπάθειες των κρατών–μελών για πλήρη εξάλειψη των διακρίσεων και πρόταξη των ατομικών και κοινωνικών ελευθεριών,

\textit{Κατανοώντας} ότι τα εκπαιδευτικά συστήματα καταβάλλουν προσπάθειες να διαμορφώσουν μια συμπεριληπτική και ανεκτική κοινωνία,

\textit{Αναγνωρίζοντας} ότι το Δικαίωμα στην Ελευθερία της Σκέψης, Συνείδησης και Θρησκείας είναι προϋπόθεση δημοκρατικής κοινωνίας,

\textit{Διαπνεόμενη} από αισθήματα προστασίας και συμπερίληψης κάθε είδους μειονότητας στα σύγχρονα κράτη,

\textit{Έχοντας} πλήρη γνώση των μεταρρυθμίσεων που επιχείρησαν ορισμένα κράτη – μέλη για την καταπολέμηση της ρητορικής μίσους,

\begin{enumerate}
  
  \item \textbf{Επιδοκιμάζει} την κινητοποίηση της Διεθνούς Αμνηστίας σε συνεργασία με τον ΟΗΕ για την αποτελεσματική προάσπιση των ανθρωπίνων δικαιωμάτων σε όλα τα κράτη μέρη.
  
  \item \textbf{Προτείνει} την δημιουργία ιστοσελίδας κρατικής ή διεθνούς εμβέλειας που θα παρέχει μεταφρασμένα άρθρα και οπτικοακουστικό υλικό εγκεκριμένα από τον Ο.Η.Ε. που θα αποσκοπούν στην ενημέρωση των πολιτών, παρέχοντας την δυνατότητα δημοσίευσης άρθρων και από τους πολίτες καθώς και την προώθηση της ιστοσελίδας μέσω διαφημίσεων και Μέσων Μαζικής Ενημέρωσης.
  
\end{enumerate}


\end{document}